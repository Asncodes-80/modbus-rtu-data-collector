\documentclass[a4paper]{article}
\usepackage{forest}
\usepackage{float}
\usepackage{geometry}
\usepackage{listings}
\usepackage{hyperref}
\usepackage{graphicx}
\usepackage{ragged2e}
\usepackage{color}
\usepackage{xepersian}
\usepackage{subfiles}
\settextfont[Scale=1]{XB Roya}

\title{مستندات مربوط به سرویس پایتون برنامه \lr{PSIM}}
\author{صنایع ارتباطی پایا | واحد تحقیقات}

\begin{document}
\maketitle

\tableofcontents

\section{آدرس‌ها و ارتباطات}

برای ورود به سیستم از اطلاعات زیر استفاده کنید:

\begin{LTR}
    \begin{table}[h]
        \centering
        \caption{آدرس شبکه‌}
            \begin{tabular}{c|cc}
                \textbf{Service} & \textbf{IP Add} & \textbf{Port \#} \\ \hline
                \lr{X-RDP} & \lr{5.201.128.78} & \lr{3389} \\ 
                \lr{SSH} & \lr{5.201.128.78} & \lr{8624} \\ 
            \end{tabular}
    \end{table}
\end{LTR}

\begin{LTR}
    \begin{table}[h]
        \centering
        \caption{کاربران تعریف شده}
            \begin{tabular}{c|cc}
                \textbf{Service} & \textbf{Username} & \textbf{Password} \\ \hline
                \lr{X-RDP} & \lr{paya} & \lr{paya@2024} \\
                \lr{SSH} & \lr{paya} & \lr{paya@2024} \\
            \end{tabular}
    \end{table}
\end{LTR}


\section{پیکربندی نرم‌افزار \lr{Zabbix} با استفاده از \lr{Docker}}

برنامه \lr{Zabbix} با اطلاعات زیر در دسترس است:

\begin{LTR}
    \begin{table}[h]
        \centering
            \begin{tabular}{ccc}
                \textbf{HTTP Address} & \textbf{Username} & \textbf{Password} \\ \hline
                \lr{localhost:8080} & \lr{Admin} & \lr{zabbix} \\
            \end{tabular}
    \end{table}
\end{LTR}

\section{مسیر فایل‌ها و منابع}

\begin{itemize}
    \item برای دسترسی به برنامه \lr{Python} به آدرس
    \lr{~/Development/modbus-rtu-data-collector} مراجعه کنید.
    \begin{itemize}
        \item برای داکیومنت برنامه به فایل \lr{README.md} مراجعه کنید.
    \end{itemize}
    \item برای دسترسی به پیکربندی \lr{docker-compose} برنامه \lr{Zabbix} به آدرس
    \lr{~/Development/zabbix-monit} مراجعه کنید.
\end{itemize}

\section{تخمین میزان حجم مورد استفاده فایل لاگ}

پس از یک روز بررسی عملکرد سرویس نوشته شده پایتون، فرمول محاسبه مقدار فضایی که
توسط فایل \lr{json} حاصل از سرویس، اشغال می‌شود مبتنی بر تعداد دستگاه‌های
\lr{PSIM} است و در طی یک سال می‌باشد:

\begin{equation}
    k = \frac{6}{T} GB/year
\end{equation}

هر دستگاه \lr{PSIM} مبتنی بر فرمول $k$، در یک سال فضایی مشخص را با واحد گیگابایت
اشغال می‌کند، که در آن $T$ زمان در واحد ثانیه می‌باشد.

برای بدست آوردن مدت زمان ذخیره‌سازی از فرمول زیر استفاده می‌کنیم:

\begin{equation}
    storage time = \frac{Total Storage}{k \times N}    
\end{equation}

که در آن $k$ از فرمول قبلی گرفته می‌شود و پارامتر $N$ تعداد دستگاه‌های \lr{PSIM}
را نشان می‌دهد.

تنظیمات اجرای سرویس توسط برنامه \lr{pm2} هر ۳ ثانیه یکبار می‌باشد.

\end{document}
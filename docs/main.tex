\documentclass[a4paper]{article}
\usepackage{forest}
\usepackage{float}
\usepackage{geometry}
\usepackage{listings}
\usepackage{hyperref}
\usepackage{graphicx}
\usepackage{ragged2e}
\usepackage{color}
\usepackage{xepersian}
\usepackage{subfiles}
\settextfont[Scale=1]{XB Roya}

\title{مستندات مربوط به سرویس پایتون برنامه \lr{PSIM}}
\author{صنایع ارتباطی پایا | واحد تحقیقات}

\begin{document}
\maketitle

پس از یک روز بررسی عملکرد سرویس نوشته شده پایتون، فرمول محاسبه مقدار فضایی که
توسط فایل \lr{json} حاصل از سرویس، اشغال می‌شود مبتنی بر تعداد دستگاه‌های
\lr{PSIM} است و در طی یک سال می‌باشد:

\begin{equation}
    k = \frac{6}{T} GB/year
\end{equation}

هر دستگاه \lr{PSIM} مبتنی بر فرمول $k$، در یک سال فضایی مشخص را با واحد گیگابایت
اشغال می‌کند، که در آن $T$ زمان در واحد ثانیه می‌باشد.

برای بدست آوردن مدت زمان ذخیره‌سازی از فرمول زیر استفاده می‌کنیم:

\begin{equation}
    storage time = \frac{Total Storage}{k \times N}    
\end{equation}

که در آن $k$ از فرمول قبلی گرفته می‌شود و پارامتر $N$ تعداد دستگاه‌های \lr{PSIM}
را نشان می‌دهد.

تنظیمات اجرای سرویس توسط برنامه \lr{pm2} هر ۳ ثانیه یکبار می‌باشد.

\end{document}